\documentclass[]{article}
\begin{document}


\title{Genomic Architecture of Highland Adaptation in Maize}

\maketitle

\section{Plant Genome Research Program (NSF)}

\subsection{Genetic architecture of highland adaptation}

\subsubsection{Questions}
	\begin{itemize}	
		\item What is the genetic basis of highland adaptation?  
		\item How much is shared between Mexico and South America? 
		\item Between maize and teosinte? 
	\end{itemize}

\subsubsection{Plan}

	\begin{itemize}
		\item QTL of high x low in Mexico and S. America
		\begin{itemize}
			\item Map at 3 sites: low Mexico (RS), high Mexico (RS), Missouri (SFG)
			\item 960 F2:3 families; 40 checks
			\item population development (SFG)
			\item DNA extraction, genotypes (SFG)
			\item Sequence parents (JRI)
			\item Mapping (SFG)
		\end{itemize}	
		
		\item Admixture mapping in \emph{mex}/\emph{parv} hybrid zone (GC \& MBH)
		\begin{itemize}
			\item 500 individuals (plant extra) from Ahuacatitlan
			\item DNA extractions (ACJ)
			\item collection trip (MBH)
			\item genotype calling (MBH \& ACJ)
			\item mapping (GC)
		\end{itemize}
		\item Phenotypes (SFG, RS, MBH):
		\begin{itemize}
				\item macro hairs
				\item flowering time
				\item tassel morphology
				\item plant height every 2 weeks \& at flowering
				\item biomass
				\item \# ears (maize), 50k weight (maize/teo), total seed weight (maize)
				\item stem/plant color
				\item germination in greenhouse (depth, temperature) (MBH)
				\item roots??
		\end{itemize}
	\end{itemize}

\subsection{Adaptation and introgression}

\subsubsection{Questions}
	\begin{itemize}	
		\item Are introgressed loci adaptive? 
		\item Does evidence of natural selection correspond to QTL? 
		\item Are highland haplotypes that are widespread in maize adapted to highland climes? 
	\end{itemize}	
		\begin{enumerate}	
		\item Increased depth/precision relative to Hufford \emph{et al.} 2013 (JR-I)

		\item Global analysis of highland haplotypes and/or \emph{F$_{ST}$} in low/high pops (Oaxaca, Ethiopia, Guatemala, etc.) (MBH \& ACJ)
	\end{enumerate}
	\item Functional characterization of QTL
	
	{\bf Questions:} What do the QTL (or selected/introgressed loci) do?
	
	\begin{enumerate}
		% T43 has Ac/Ds
		\item PT x T43 NIL population development	
		\item Fine map pigmentation
		\item 4 parents + mex + parv RNAseq time series
		\item Allelic series at some other QTL in T43, B73 and CML457
			\begin{itemize}
			\item 10 parents (4 from F2:3, mexicana, PT, 2 more lowland, 2 more highland)
			\end{itemize}
		% RS to check T43 in highland sites
		% MBH to check T43 and CML457 in Iowa
	\end{enumerate}	

\end{enumerate}

\section*{Catalyzing New International Collaborations (NSF)}
Due Date:  Rolling acceptance. 

\begin{enumerate}

\item Need to submit request for support to NSF Plant Genome prior to submitting grant.

\item Funds two meetings to discuss grants/preliminary data, two seasons of line development, sequencing of mapping population parents and/or GBS of populations.

\end{enumerate}

\end{document}
