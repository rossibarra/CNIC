\documentclass[]{article}
\begin{document}


\title{The Genomic Basis of Local Adaptation in Maize}

\maketitle

\section*{Catalyzing New International Collaborations (NSF)}
Due Date:  Rolling acceptance. 

\begin{enumerate}

\item Need to submit request for support to NSF Plant Genome prior to submitting grant.

\item Funds two meetings to discuss grants/preliminary data, two seasons of line development, sequencing of mapping population parents and/or GBS of populations.

\end{enumerate}

\section*{Plant Genome Research Program (NSF)}
Due Date: March 2014. 

\subsection*{Components}
\begin{enumerate}
	\item Genetic architecture of highland adaptation
	
	{\bf Questions:} What is the genetic basis of highland adaptation?  How much is shared between Mexico and South America? Between maize and teosinte? 
	\begin{enumerate}
		\item QTL of high x low in Mexico and S. America (SF-G (\& RS?))
		
		SF-G on S. America and Mexico? 
		
		Grow both high and low? Who does which?

		\item \emph{parviglumis} x \emph{mexicana} cross
		
		Doebley has seed of TC1: (\emph{mex} x \emph{parv}) x A158 (who?) 
		\item Admixture mapping in \emph{mex}/\emph{parv} hybrid zone (GC \& MBH)

		GC on analysis, new theory (better precision mapping?)
		
		MBH on phenotyping and genotyping
	\end{enumerate}
	\item Adaptation and introgression
	
	{\bf Questions:} Are introgressed loci adaptive? Does evidence of natural selection correspond to QTL? Are highland haplotypes that are widespread in maize adapted to highland climes?
	\begin{enumerate}	
		\item Increased depth/precision relative to Hufford \emph{et al.} 2013 (JR-I)

		\item Global analysis of highland haplotypes and/or \emph{F$_{ST}$} in low/high pops (Oaxaca, Ethiopia, Guatemala, etc.) (MBH \& ACJ)
	\end{enumerate}
	\item Functional characterization of QTL
	
	{\bf Questions:} What do the QTL (or selected/introgressed loci) do?
	
	Something here about RS's NILs or other introgression pops. (RS)

\end{enumerate}

\section*{Outstanding questions}
\begin{enumerate}
	\item Where will growouts take place?  Puerto Rico (SF-G), Mexico, Hawaii?
	\item Do we include highland and lowland sites for all phenotyping?
	\item What will South American cross include? 
	\item What needs to be sequence/GBSed for CNIC and Plant Genome grant? 
	\begin{enumerate}
		\item Outbred highland landraces X lowland Doebley inbred (SF-G will set this up summer 2013)? 
		\item DH highland landrace X lowland Doebley inbred (MBH growth chamber w/ seed from SF-G)?
		\item Highland Murray inbred X lowland Doebley inbred
	\end{enumerate}
	\item Do we try to drill down on a small set of genes?
	\item Should we include Southwest US material (there are both highland and lowland accessions)?
	\item Do we include \emph{parviglumis} x \emph{mexicana} cross and if so who does it?
	\item What are broader impacts/outreach? Continue US-Mexico student exchange program currently implemented in Maize Centromere NSF-Plant Genome Grant?
	\item Monthly and annual meetings?
	\item Students/pdocs/shared field work?
	\item Do we measure other (harder) phenotypes?
		\begin{enumerate}
		\item Root chilling
		\item UV sensitivity
		\item Germination
		\end{enumerate}
	\item Can we leverage SEED data being generated by CIMMYT (GBS of many, many landraces)?
	\item Basic budget (personnel, sequencing, field facilities)?
	\item Basic timeline?

\end{enumerate}


\end{document}
